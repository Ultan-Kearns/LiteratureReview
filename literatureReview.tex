\documentclass{report}
\usepackage[utf8]{inputenc}
\usepackage{graphicx}
\usepackage{listings}
\usepackage{hyperref}
\usepackage{amsmath}
\title{Literature Review - An Analysis of P vs N}
\author{Ultan Kearns}
\date{\today}
\begin{document}
\maketitle
\tableofcontents
\chapter{Introduction}
\section{P vs NP Problem - A Brief Introduction}
The P vs NP problem is a very famous problem in computer science.  The problem
can basically be described as following: if a computer is able to check the answer
to a problem can that computer actually solve said problem?(In polynomial time)\cite{wiki}  P problems can be
solved faster by computers than NP problems and are termed as "easy" problems, NP problems
are "easy" for a computer to check but are not "easy" for a computer to solve.\cite{wiki}
If P $!=$ NP($!=$ means not equal) then problems in NP are harder to compute than to verify
this means they could not be solved in polynomial time but could be checked in polynomial time.  We will explore this more in later chapters.
\section{An Explanation of Polynomial Time}
Polynomial time is a term that is applied to algorithms if the number of steps
to complete said algorithm for a given input is $O(n^k)$ where $k$ is any non-negative
integer and where $n$ is the algorithmic complexity of the input to the algorithm \cite{polynomial}
Everyday calculations such as all basic arithmetic and calculating digits of $\pi$ and $e$ are said
to be done in polynomial time by a computer\cite{polynomial}.
\section{Analysis of Project}
In this project I will compare various different papers by various authors and
analyze and review their works.  I will also cite various journals and professionals
who have extensive experience in this area and with the P vs NP problem.
\section{Outline of Chapters}
\begin{itemize}
  \item Chapter 1 - Introduction - This chapter introduces the problem and summarizes it
  \item  Chapter 2 - P vs NP history - This chapter discusses the history of P vs NP and will give examples of such a problem
  \item Chapter 3 - Why is P vs NP important - An analysis of the importance of this question and the ramifications it could bring to computer science if proven or disproven
  \item Chapter 4 - Conclusion - This chapter will discuss the information I have gathered while researching this topic and the future of the P vs NP problem
\end{itemize}
\chapter{P vs NP - A History \& Analysis}
\section{Brute Force Explained}
To understand the P vs NP problem we must first understand algorithmic complexity and the undesirability of the brute force methodology in algorithms.
Brute force refers to an algorithm which is exhaustive and searches all possibilities before arriving at an answer, this is undesirable for many reasons such as: it has a high computing time, demands more computing power and can take up large amounts of memory when compared to non brute force algorithms. The first "lucid" account, according to the author, of brute force search appearing in western literature according to Sisper was by a researcher named Edmonds\cite{HistoryOfPVsNP}, he explained the problem of brute force search in the algorithmic method.  The author also mentions that Von Neumann(famous computer scientist\cite{Neumann}) recognised the undesirability of brute force search in a computer program. There are several earlier papers which the author mentions in the chapter P and NP. He mentions papers by Rabin\cite{ResearchPaperRabin},Hartmanis \& Stearns\cite{ResearchPaperHartmanis},Blum\cite{ResearchPaperBlum} which discussed the proposition of measuring the complexity of an algorithm in the number of steps required to solve it. The attempt to measure algorithmetic complexity gave rise to the P vs NP problem.
\section{Origins of P vs NP}
The P vs NP question originated due to developments in mathematical and electronic progressions in the mid twentieth century\cite{HistoryOfPVsNP}. It's now a prominent question in the field of computer science and one of the Millennium Problems\cite{Millennium}, these are a list of problems deemed most important by the Clay Mathematical Institute, which is a private operating foundation that aims to disseminate mathematical knowledge\cite{AboutMillennium}.  In the first half of the 20th century mathematicians were working on the formal systems of mathematics\cite{ResearchPaperFormalizeMathematics}, this led to appearances of problems in set theory such as Russell's Paradox\cite{RussellParadox} and Gödel's incompleteness theorem\cite{Godel}. Such work on formalization led to the growing realization that certain algorithms are unsolvable\cite{HistoryOfPVsNP}.
\section{P vs NP Real World Example}
To use the example Sisper used in his research paper\cite{HistoryOfPVsNP}, imagine that you are searching for an object in a wide range of possibilities.  Computers could expedite your search but in the case of finding the object in an extremely large space would require exhaustive searching even on the fastest machines.  The P vs NP question in this example can be reduced down to the following: "Could there be some method to perform such an exhaustive search without having to resort to brute force?".
\section{Analysis of Current State of Problem and Research}
There has been much debate and speculation from researchers if the P vs NP problem is solvable, I have come across articles that have claimed to solve the P vs NP problem such as\cite{PVsNPSolved} which was written by two students from The Rensselaer AI & Reasoning (RAIR) LabRensselaer Polytechnic Institute (RPI), and papers which claim to prove it is not solvable\cite{P!=NP}. I will review both of these papers to analyze the cureent state of the problem and to critically evaluate the subject matter.
\subsection{P = NP?}
\subsection{P != NP?}
\section{Analysis of Papers Reviewed}
In this paper titled: The History and Status of the P versus NP Question Sisper outlines the history of the P vs NP problem.  He gives real world examples such as the one referenced earlier in this chapter in the section titled: P vs NP Real World Example.
\chapter{Importance of P vs NP}
\chapter{Conclusion}

\bibliographystyle{ieeetr}
\bibliography{bibliography}
\end{document}
