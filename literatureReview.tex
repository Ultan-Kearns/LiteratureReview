\documentclass{report}
\usepackage[utf8]{inputenc}
\usepackage{graphicx}
\usepackage{listings}
\usepackage{hyperref}
\usepackage{amsmath}
\title{Literature Review - An Analysis of P vs N}
\author{Ultan Kearns}
\date{\today}
\begin{document}
\maketitle
\tableofcontents
\chapter{Introduction}
\section{What is P vs NP problem}
The P vs NP problem is a very famous problem in computer science.  The problem
can basically be described as following: if a computer is able to check the answer
to a problem can that computer actually solve said problem?(In polynomial time)\cite{wiki}  P problems can be
solved faster by computers than NP problems and are termed as "easy" problems, NP problems
are "easy" for a computer to check but are not "easy" for a computer to solve.\cite{wiki}
If P $!=$ NP($!=$ means not equal) then problems in NP are harder to compute than to verify
this means they could not be solved in polynomial time but could be checked in polynomial time.
\section{What is polynomial time?}
Polynomial time is a term that is applied to algorithms if the number of steps
to complete said algorithm for a given input is $O(n^k)$ where $k$ is any non-negative
integer and where $n$ is the algorithmic complexity of the input to the algorithm \cite{polynomial}
Everyday calculations such as all basic arithmetic and calculating digits of $\pi$ and $e$ are said
to be done in polynomial time by a computer\cite{polynomial}.
\section{Analysis of project}
In this project I will compare various different papers by various authors and
analyze and review their works.  I will also cite various journals and professionals
who have extensive experience in this area and with the P vs NP problem.
\section{Outline of chapters}
\begin{itemize}
  \item Chapter 1 - Introduction - Here I introduce the problem and explain a bit
  about it
  \item  Chapter 2 - P vs NP history - Here I will discuss how the problem originated and its history
  \item Chapter 3 - Why is P vs NP important - In this chapter I will discuss the importance of this problems
  and the effect it will have on computer science if it is proven or disproven
  \item Chapter 4 - Background and experience of researchers - In this chapter I'll discuss the background of
  each researcher whose papers I have chosen to review
  \item Chapter 5 - In this chapter I will review the works of the researchers which is the crux of this assignment
\end{itemize}
\chapter{P vs NP History}
The P vs NP question originated due to developments in mathematical and electronic progressions in the mid twentieth century\cite{HistoryOfPVsNP}. It's now a prominent question in the field of computer science and one of the Millennium Problems\cite{Millennium}, these are a list of problems deemed most important by the Clay Mathematical Institute, which is a private operating foundation that aims to disseminate mathematical knowledge\cite{AboutMillennium}.  In the first half of the 20th century mathematicians were working on the formal systems of mathematics\cite{ResearchPaperFormalizeMathematics}, this led to appearances of problems in set theory such as Russell's Paradox\cite{RussellParadox} and Gödel's incompleteness theorem\cite{Godel}.
\section{Formulation of P vs NP}

\chapter{Why is The P vs NP Problem Important?}
\chapter{Background \& Experience of Researchers}
\chapter{Analysis \& review of Papers}

\bibliographystyle{ieeetr}
\bibliography{bibliography}
\end{document}
