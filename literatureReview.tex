\documentclass{report}
\usepackage[utf8]{inputenc}
\usepackage{graphicx}
\usepackage{listings}
\usepackage{hyperref}
\usepackage{amsmath}
\title{Literature Review - An Analysis of P vs N}
\author{Ultan Kearns}
\date{\today}
\begin{document}
\maketitle
\tableofcontents
\chapter{Introduction}
\section{What is P vs NP problem}
The P vs NP problem is a very famous problem in computer science.  The problem
can basically be described as following: if a computer is able to check the answer
to a problem can that computer actually solve said problem?\cite{wiki}  P problems can be
solved faster by computers than NP problems and are termed as "easy" problems, NP problems
are "easy" for a computer to check but are not "easy" for a computer to solve.\cite{wiki}
If P != NP(!= means not equal) then problems in NP are harder to compute than to verify
this means they could not be solved in polynomial time but could be checked in polynomial time.
\section{What is polynomial time?}
Polynomial time is a term that is applied to algorithms if the number of steps
to complete said algorithm for a given input is $O(n^k)$ where $k$ is any non-negative
integer and where $n$ is the algorithmic complexity of the input to the algorithm \cite{polynomial}
Everyday calculations such as all basic arithmetic and calculating digits of $\pi$ and $e$ are said
to be done in polynomial time by a computer\cite{polynomial}.
\section{Analysis of project}
\section{Outline of chapters}
\chapter{Researching the problem}
\chapter{Analysis of Papers}
\bibliographystyle{ieeetr}
\bibliography{bibliography}
\end{document}
